\PassOptionsToPackage{unicode=true}{hyperref} % options for packages loaded elsewhere
\PassOptionsToPackage{hyphens}{url}
%
\documentclass[]{article}
\usepackage{lmodern}
\usepackage{amssymb,amsmath}
\usepackage{ifxetex,ifluatex}
\usepackage{fixltx2e} % provides \textsubscript
\ifnum 0\ifxetex 1\fi\ifluatex 1\fi=0 % if pdftex
  \usepackage[T1]{fontenc}
  \usepackage[utf8]{inputenc}
  \usepackage{textcomp} % provides euro and other symbols
\else % if luatex or xelatex
  \usepackage{unicode-math}
  \defaultfontfeatures{Ligatures=TeX,Scale=MatchLowercase}
\fi
% use upquote if available, for straight quotes in verbatim environments
\IfFileExists{upquote.sty}{\usepackage{upquote}}{}
% use microtype if available
\IfFileExists{microtype.sty}{%
\usepackage[]{microtype}
\UseMicrotypeSet[protrusion]{basicmath} % disable protrusion for tt fonts
}{}
\IfFileExists{parskip.sty}{%
\usepackage{parskip}
}{% else
\setlength{\parindent}{0pt}
\setlength{\parskip}{6pt plus 2pt minus 1pt}
}
\usepackage{hyperref}
\hypersetup{
            pdftitle={Looking for evidence of a high burden of COVID-19 in the United States from influenza-like illness data},
            pdfauthor={Nicholas G. Reich, Evan L. Ray, Graham C. Gibson, Estee Cramer, Caitlin M. Rivers},
            pdfborder={0 0 0},
            breaklinks=true}
\urlstyle{same}  % don't use monospace font for urls
\usepackage[margin=1in]{geometry}
\usepackage{graphicx,grffile}
\makeatletter
\def\maxwidth{\ifdim\Gin@nat@width>\linewidth\linewidth\else\Gin@nat@width\fi}
\def\maxheight{\ifdim\Gin@nat@height>\textheight\textheight\else\Gin@nat@height\fi}
\makeatother
% Scale images if necessary, so that they will not overflow the page
% margins by default, and it is still possible to overwrite the defaults
% using explicit options in \includegraphics[width, height, ...]{}
\setkeys{Gin}{width=\maxwidth,height=\maxheight,keepaspectratio}
\setlength{\emergencystretch}{3em}  % prevent overfull lines
\providecommand{\tightlist}{%
  \setlength{\itemsep}{0pt}\setlength{\parskip}{0pt}}
\setcounter{secnumdepth}{0}
% Redefines (sub)paragraphs to behave more like sections
\ifx\paragraph\undefined\else
\let\oldparagraph\paragraph
\renewcommand{\paragraph}[1]{\oldparagraph{#1}\mbox{}}
\fi
\ifx\subparagraph\undefined\else
\let\oldsubparagraph\subparagraph
\renewcommand{\subparagraph}[1]{\oldsubparagraph{#1}\mbox{}}
\fi

% set default figure placement to htbp
\makeatletter
\def\fps@figure{htbp}
\makeatother


\title{Looking for evidence of a high burden of COVID-19 in the United States
from influenza-like illness data}
\author{Nicholas G. Reich, Evan L. Ray, Graham C. Gibson, Estee Cramer, Caitlin
M. Rivers}
\date{2020-03-20 20:58:33 CET}

\begin{document}
\maketitle

\hypertarget{introduction}{%
\subsection{Introduction}\label{introduction}}

In December 2019, an outbreak of a novel, the SARS-CoV-2 coronavirus was
detected in Wuhan, China. In the intervening weeks, case counts have
grown substantially. As of this writing, there are over 234,000
confirmed cases globally and at least 9,840 deaths from what is
currently named COVID-19 {[}1{]}. It is now understood that the virus
transmits efficiently from person to person, with R0 estimates above 2
and perhaps as high as 3.7 {[}2, 3{]}.

Community transmission is now ongoing in many locations in the United
States. Emerging phylogenetic data suggest that sequenced cases to date
in the United States and globally share a common ancestor between
mid-November and mid-December 2019. However, due to delays in making
widespread diagnostic testing available, the burden of COVID-19 in the
United States is not well understood. As an effort to understand trends
in people seeking care for respiratory symptoms, we analyze publicly
available data on influenza-like illness in the US. Specifically, we
compare the proportion of weighted influenza like illness (wILI) that
tests negative for influenza during the 2019-2020 flu season to trends
from previous seasons. If it were the case that SARS-CoV-2 were causing
widespread disease in the United States, we might expect to see in
recent weeks a higher fraction of ILI specimens that test negative for
influenza compared to the same time in past seasons.

\hypertarget{methods}{%
\subsection{Methods}\label{methods}}

\hypertarget{data}{%
\paragraph{Data}\label{data}}

We downloaded publicly available ILINet and WHO-NREVSS data for US
Health and Human Services (HHS) regions (Figure 1) and states.

\begin{figure}
\centering
\includegraphics{ili-labtest-report_files/figure-latex/hhs-regions-map-1.pdf}
\caption{\label{fig:hhs-regions-map}US HHS Regions are made up of groups
of states.}
\end{figure}

From the ILINet dataset, we downloaded weighted influenza-like illness
(wILI), which measures the percentage of doctor's office visits at
sentinel providers that had the primary complaint of fever plus an
additional influenza-like symptom (cough, and/or sore throat). For the
WHO-NREVSS data, we obtained the total number of specimens tested by
participating clinical laboratories, as well as the percent of those
specimens that tested positive for influenza. We used 23 seasons of data
for HHS regions, beginning with the 1997/1998 season, and 10 seasons of
data for states, beginning with the 2010/2011 season. All data sources
are available at the weekly time-scale, defined as using the MMWR week
standard used by the CDC.

The code used to produce this report is available on GitHub at
\url{https://github.com/reichlab/ncov}.

\hypertarget{influenza-like-illness-not-attributable-to-influenza}{%
\paragraph{Influenza-like illness not attributable to
influenza}\label{influenza-like-illness-not-attributable-to-influenza}}

One possible measure of influenza illness not attributable to influenza
(ILI-) can be calculated as follows:

\[\text{ILI-} = (1 - \text{proportion of tests positive for influenza}) \times \text{wILI}\]

It is important to note that reported wILI can vary substantially due to
differences in the types of health care providers reporting into ILINet.
Therefore, some increases in reported wILI from one season to another
may be driven in part by changes in provider type make up. An
approximate way to adjust for this is by dividing reported wILI by the
baseline for a given region and season. Baselines for HHS regions are
provided by the CDC. Baselines for states are calculated as the average
of the first two weekly ILI observations for a given season, thinking
that this adjusts for any systematic adjustments to the provider mix in
each season. These baselines enable the following calculation of a
\textbf{r}elative ILI-.

\[\text{rILI-} = (1 - \text{proportion of tests positive for influenza}) \times \frac{\text{wILI}}{\text{baseline level for ILI}}\]

\hypertarget{measuring-anomalies-in-ili--during-a-season}{%
\paragraph{Measuring anomalies in ILI- during a
season}\label{measuring-anomalies-in-ili--during-a-season}}

We developed a metric to measure the degree to which a given ILI-
observation is significantly higher or lower than expectation, based on
past trends at similar times of the year. For each region and
season-week, we averaged observations from the past seasons (22 seasons
for regions, 9 for states) for the given season week and one season week
on either side and calculated the standard deviation based on these same
observations. We then computed ``z-scores'' as the number of standard
deviations above or below the average a particular rILI- observation is:
\[\text{Z} =  \frac{\text{rILI-} - \overline{\text{rILI-}}}{sd{\text{rILI-}}}\]

\hypertarget{results-discussion}{%
\subsection{Results \& Discussion}\label{results-discussion}}

This report uses data downloaded on March 20, 2020, with data reported
through March 14, 2020.

\hypertarget{regional-level-analyses}{%
\subsubsection{Regional-level analyses}\label{regional-level-analyses}}

We plotted ILI- and rILI- as a function of the week within each flu
season and stratified by region (Figure 2). Additionally, we plotted the
2019/2020 Z-scores for all regions as a function of week of season
(Figure 3).

In the last weeks of 2019 and first weeks of 2020, the observations of
ILI burden due to non-influenza pathogens (rILI-) have been, relative to
what has been observed in the past 22 seasons, above the seasonal
average. More recently, the measures of rILI- also have risen to more
than 3 standard deviations above the average in some regions (Figure 3).

These results suggest that there may be enough COVID-19 circulating in
the United States to be detectable in the influenza-like illness
surveillance system. However, it is hard to determine this conclusively,
as we have not performed an exhaustive analysis about what other
pathogens were or were not ciruclating in those past seasons. Also,
media attention could also drive more individuals with mild
influenza-like illness symptoms to seek care than usual even in the
absence of widespread COVID-19 transmission in the US. If these
additional individuals seeking care were more likely to have an illness
not caused by influenza, then this could also drive up the rILI- metric.

\begin{figure}
\centering
\includegraphics{ili-labtest-report_files/figure-latex/all-region-plot-ILI-1.pdf}
\caption{\label{fig:all-region-plot}US HHS Regions plots showing rILI-
values since the 1997/1998 season (grey lines) and the 2019/2020 season
(dark black line). The line highlighted in red is the 2009/2010 H1N1
pandemic season. The dates on the x-axis correspond to the dates for the
2019/2020 season, with previous seasons lined up approximately the same
time by week. The vertical dashed line shows the date at which this plot
was generated. The small gap between the current season's data and the
line indicates the lag in ILI reporting, typically one week.}
\end{figure}

\begin{figure}
\centering
\includegraphics{ili-labtest-report_files/figure-latex/std-dev-analysis-1.pdf}
\caption{Figure showing Z scores by week for each HHS region. Tiles with
a dark black outline indicate locations where the observed rILI- was
higher in that week of the season than had ever been observed in the
last 22 seasons.}
\end{figure}

\hypertarget{state-level-analyses}{%
\subsubsection{State-level analyses}\label{state-level-analyses}}

The z-score calculations show six states with systematically higher than
average observations (Figure 4). We think many of these may be spurious,
due to other systematic differences in reporting for this season.
Although at this time we do not have specific data to support or refute
this hypothesis.

We note an increased signal in Missouri, Pennsylvania, Arizona,
California, Massachusetts, and West Virginia (Figure 5). California and
Washington are known to have community transmission of SARS-CoV-2, but
from this analysis alone we are not able to determine whether that is
contributing to the observed signal in rILI-.

Missouri has recorded relatively few cases of COVID-19 to date. We note
that only 9 providers contributed data to the ILI surveillance network
in the most recent week of data compared to 11 to 13 in earlier weeks.
Provider type (e.g.~pediatrician, hospital) is known to influence ILI
data. If there are systematic differences in the population seeking care
at different providers, that could explain the recently observed spike.
We have been in touch with the Missouri Department of Health. They are
aware of our finding and have confirmed that they are carefully
monitoring for SARS-CoV-2 activity.

In recent weeks, several states have had an uptick in ILI while
simultaneously experiencing a decrease in the percentage of influenza
tests that are positive (Figure 6). These trends are also suggestive of
widespread circulation of an ILI-causing pathogen that is not influenza.

Syndromic surveillance data is best used to guide further investigation,
not as a definitive source. Additional study is needed to understand the
burden of SARS-CoV-2 across U.S. states. However, we believe that this
analysis could be used to guide and target areas for further
investigation

\begin{figure}
\centering
\includegraphics{ili-labtest-report_files/figure-latex/calc-avg-sd-all-states-1.pdf}
\caption{Figure showing Z scores by week for each state in the US that
has collected and tested more than 5000 specimens for influenza during
the 2019/2020 season. Tiles with a dark black outline indicate locations
where the observed rILI- was higher in that week of the season than had
ever been observed in the last 22 seasons. Some states have no reported
tests for a given week and so the rILI- is missing for that week. States
are sorted by z-score in the most recent week, with highest scores at
the bottom (states with missing z-scores for the most recent week are at
the top).}
\end{figure}

\begin{figure}
\centering
\includegraphics{ili-labtest-report_files/figure-latex/get-sd-data-1.pdf}
\caption{Selected individual state-level plots showing the proportion of
ILI not due to influenza over the current 2019/2020 season (dark black
line) and past 9 seasons, starting with 2010/2011.}
\end{figure}

\begin{figure}
\centering
\includegraphics{ili-labtest-report_files/figure-latex/state-unweighted-ili-by-pos-neg-1.pdf}
\caption{Figure showing inferred breakdown of unweighted ILI into ILI
that is positive and negative for influenza for states with greater than
5,000 influenza tests run in the 2019/2020 season.Unweighted ILI
positive for influenza is calculated as the product of Unweighted ILI
and the proportion of laboratory tests positive for influenza.}
\end{figure}

\hypertarget{works-cited}{%
\subsection{Works Cited}\label{works-cited}}

{[}1{]}
\url{https://www.who.int/docs/default-source/coronaviruse/situation-reports/20200319-sitrep-59-covid-19.pdf?sfvrsn=c3dcdef9_2}

{[}2{]} Yang, Y., Lu, Q., Liu, M., Wang, Y., Zhang, A., Jalali, N.,
Dean, N., Longini, I., Halloran, M. E., Xu, B., Zhang, X., Wang, L.,
Liu, W., \& Fang, L. (2020). Epidemiological and clinical features of
the 2019 novel coronavirus outbreak in China. MedRxiv,
2020.02.10.20021675. \url{https://doi.org/10.1101/2020.02.10.20021675}

{[}3{]} Imai, N., Cori, A., Dorigatti, I., Baguelin, M., Donnelly, C.
A., \& Riley, S. (n.d.). Report 3: Transmissibility of 2019-nCoV.
\url{https://www.imperial.ac.uk/media/imperial-college/medicine/sph/ide/gida-fellowships/Imperial-2019-nCoV-transmissibility.pdf}.

\clearpage

\hypertarget{changelog}{%
\subsection{Changelog}\label{changelog}}

20 March 2020: Added figure to state level analysis showing estimated
breakdown of unweighted ILI into unweighted ILI positive for or negative
for influenza. Updated for new ILI data, and reframed the introduction
to acknowledge community transmission in the US and discussion to
accomodate new results.

13 March 2020: updated for new ILI data, including interpretation about
Missouri.

6 March 2020: updated for new ILI data, genomic commentary, some state
figures.

2 March 2020: Added state-level analysis, HHS region map, z-score code
and figure.

29 February 2020: updated for new ILI data. Minor rephrasing in intro.

21 February 2020: updated for new ILI data.

16 February 2020: updated to revise name of COVID-19, updated case
counts and ILINet data, added citations and revised statements about R0.

2 February 2020: Updated to include new ILINet data released on Friday,
Jan 31.

26 January 2020: Although our overall assessment has not changed and our
analysis has not been updated, we have updated the discussion to better
convey the level of uncertainty in our analysis. We also added a heavier
line for the 2019/2020 season in the figures.

25 January 2020: First version of report released.

\end{document}
